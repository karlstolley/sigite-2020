\documentclass[sigconf]{acmart}
\usepackage[utf8]{inputenc}

\AtBeginDocument{%
  \providecommand\BibTeX{{%
    \normalfont B\kern-0.5em{\scshape i\kern-0.25em b}\kern-0.8em\TeX}}}

\setcopyright{rightsretained}
\settopmatter{printacmref=true}

\begin{document}
\fancyhead{}

\title{Enhancing CSS Instruction with Objective Typography}

\author{Karl Stolley}
\email{kstolley@iit.edu}
\orcid{1234-5678-9012}
\affiliation{%
  \institution{Illinois Institute of Technology}
  \streetaddress{10 West 35th Street}
  \city{Chicago}
  \state{IL}
  \country{USA}
  \postcode{60616}
}

\copyrightyear{2020}
\acmYear{2020}
\acmConference[SIGITE '20]{The 21st Annual Conference on Information Technology Education}{October 7--9, 2020}{Virtual Event, USA}
\acmBooktitle{The 21st Annual Conference on Information Technology Education (SIGITE '20), October 7--9, 2020, Virtual Event, USA}
\acmDOI{10.1145/3368308.3415439}
\acmISBN{978-1-4503-7045-5/20/10}

\begin{abstract}
	Classroom instruction in Cascading Style Sheets (CSS) can be made more rigorous and systematic when anchored by established, rule-governed typesetting principles from the International Typographic Style. This poster, which is prepared as an interactive website in the spirit of this year’s online conference, outlines an instructional approach to CSS that requires students to apply a core set of typographic principles to a manageable, human-scaled subset of CSS properties. In this approach, students  formally express a set of abstract but reasoned design principles as a collection of CSS style declarations.
\end{abstract}

\begin{CCSXML}
<ccs2012>
<concept>
<concept_id>10003456.10003457.10003527.10003531.10003535</concept_id>
<concept_desc>Social and professional topics~Information technology education</concept_desc>
<concept_significance>500</concept_significance>
</concept>
<concept>
<concept_id>10003456.10003457.10003527.10003528</concept_id>
<concept_desc>Social and professional topics~Computational thinking</concept_desc>
<concept_significance>300</concept_significance>
</concept>
</ccs2012>
\end{CCSXML}

\ccsdesc[500]{Social and professional topics~Information technology education}
\ccsdesc[300]{Social and professional topics~Computational thinking}

\keywords{Cascading Style Sheets, typography, web design, curriculum}


\maketitle

\section{Problem}
HTML and JavaScript are governed by an instructionally convenient structural formalism that CSS lacks. For example, HTML demands a sensible source order and is only valid if it includes certain structures, such as the HTML DOCTYPE. JavaScript, as a scripting language without the forgiving error-recovery of HTML, establishes its own strict internal order and rules.

By contrast, CSS is a comparative free-for-all: problems of inheritance and selector specificity aside, a stylesheet may contain any number of selectors and style declarations, presented in any order. Malformed, redundant, or inapplicable lines of CSS are simply ignored and effectively discarded by web browsers, whose internal stylesheets further complicate effective instruction.

Classroom instruction in CSS reflects the language's inherent disorder \cite{pw:learning}, and in the worst cases devolves into an unstructured sight-seeing tour of CSS properties and their effects: “This is how to make text pink,” “This is how to set a background color,” “This is how to remove an underline.” But the sight-seeing approach and its variants—walk-throughs of related groups of CSS properties, source-level inspection of stylesheets on existing pages—showcase the design features that CSS can control at the expense of what students need: a CSS development approach as reasoned, methodical, and rule-governed as development in HTML and JavaScript.

\section{Solutions}

With few rules of its own, CSS instruction can be enhanced by subjecting it to external rules and constraints. This instructional approach focuses on setting text and interface elements according to the rules of “objective and functional typography” \cite{mb:grid}. While adherence to those rules often yields a pleasing aesthetic result, aesthetics are incidental to the instructional method.

Previewing work in a browser's responsive design view set to a narrow mobile scale directs students' focus on establishing a vertical rhythm for a selection of HTML text on just a single axis. Students learn to experiment with readable text sizes and leading, initially using absolute units (pixels, points) just two CSS properties (\verb|font-size| and \verb|line-height|).

That work establishes two anchor values for setting type on the page: the base font size, which can be resized for headings and other elements using values found in the traditional typographers scale or a modular scale \cite{brown:modular}, and the line height, which can be presented in multiples or fractions of the base value for a consistent, mathematically precise setting. Students then learn to convert absolute units to accessible relative units (ems, rems).

\section{Conclusion}
With its emphasis on applying external rules to a small set of interdependent CSS properties, this instructional approach provides students with a strong foundation for more sophisticated page design. That includes commanding the complex sets of properties in the CSS specification's newer layout modules, especially CSS Flexible Boxes and CSS Grid.

This approach has been refined in the classroom over a four-year period, but the development of an instrument, such as an interactive game \cite{kim:understanding}, is needed to assess the approach's effectiveness.

\bibliographystyle{ACM-Reference-Format}
\bibliography{stolley.bib}

\end{document}
